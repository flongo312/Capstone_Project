\section{Literature Review}

\subsection{Portfolio Selection by Harry Markowitz (1952)}

\subsubsection{Key Takeaways}
In "Portfolio Selection," Harry Markowitz introduced the concept of Modern Portfolio Theory (MPT), which fundamentally changed the way investors approach portfolio construction. Markowitz emphasized the trade-off between risk and return, proposing that investors should diversify their investments across various assets to minimize risk without sacrificing expected returns. The main contribution of this paper is the efficient frontier, which represents the set of portfolios that offer the highest expected return for a given level of risk. The efficient frontier helps investors identify the optimal asset allocation that maximizes returns for a specific risk level.

\begin{quote}
\textit{"An investor should diversify to achieve maximum return for a given level of risk, leading to the creation of the efficient frontier."} \citep{markowitz1952portfolio}.
\end{quote}

Markowitz's work also introduced the concept of mean-variance optimization, where the expected return of a portfolio is calculated as the weighted sum of the expected returns of individual assets, and the risk is measured as the variance of portfolio returns. This approach allows investors to construct portfolios that either maximize expected return for a given level of risk or minimize risk for a given level of expected return.

\subsubsection{Study Implications}
Markowitz's findings are fundamental to this study as they provide the basis for constructing diversified investment portfolios aimed at accumulating down payments for housing. By applying the principles of MPT, my study seeks to identify optimal asset allocations that maximize returns while managing risk over different investment horizons.

\subsubsection{Critiques}
Despite its groundbreaking contributions, MPT has faced criticism for its reliance on historical data to estimate future returns and risks, which may not always be accurate. Additionally, the assumption of normally distributed returns has been questioned, as real-world financial returns often exhibit skewness and kurtosis. Furthermore, MPT also does not account for extreme events and tail risks, which can significantly impact portfolio performance.





\subsection{Capital Asset Prices: A Theory of Market Equilibrium under Conditions of Risk by William F. Sharpe (1964)}

\subsubsection{Key Takeaways}
In his paper, William F. Sharpe introduced the Capital Asset Pricing Model (CAPM), which provides a framework to determine the expected return of an asset based on its systematic risk, as measured by beta. The model states that the expected return on an asset is a function of the risk-free rate, the asset's beta, and the market risk premium. Sharpe also developed the Sharpe Ratio, a measure of risk-adjusted return, which helps investors understand how much excess return they are receiving for the extra volatility endured.

\begin{quote}
\textit{"The expected return on a security is linearly related to its beta, reflecting its sensitivity to market movements."} \citep{sharpe1964capital}.
\end{quote}


\subsubsection{Study Implications}
Sharpe's CAPM and Sharpe Ratio are crucial for assessing the risk and return profiles of individual securities within the portfolios analyzed in this study. By evaluating the beta and risk-adjusted returns, this study can better understand the performance of different assets and optimize the investment strategies accordingly. The CAPM provides a systematic approach to estimating the expected returns of assets based on their systematic risk, while the Sharpe Ratio helps in comparing the performance of investments on a risk-adjusted basis.

\subsubsection{Critiques}
The CAPM has been criticized for its assumption of a single-period investment horizon and the use of a single market index to represent the entire market. Additionally, the model's reliance on beta as the sole measure of risk is a problem, as it does not account for other factors that may influence asset returns. Furthermore, the CAPM's assumption of a frictionless market and homogeneous expectations among investors may not hold true in real-world scenarios.






\subsection{Options: A Monte Carlo Approach by Phelim P. Boyle (1977)}

\subsubsection{Key Takeaways}
Phelim P. Boyle's paper introduced the use of Monte Carlo methods for option pricing, providing a robust tool to model the uncertainty and variability in financial investments. Monte Carlo simulations generate a large number of random samples from the probability distributions of asset returns, allowing investors to assess the impact of risk and uncertainty on their portfolios. In the paper, Boyle describes the Monte Carlo simulation process which involves generating random returns based on historical data and iterating this process to build a distribution of potential outcomes. He states that by running multiple simulations, investors can estimate the expected value and variability of the investment portfolio, providing insights into the likelihood of achieving their financial goals.

\begin{quote}
\textit{"Monte Carlo simulations enable the modeling of complex financial instruments and the assessment of risk and return in uncertain environments."} \citep{boyle1977options}.
\end{quote}

\subsubsection{Study Implications}
Boyle's Monte Carlo methods are integral to this study for forecasting the performance of optimized portfolios under various market conditions. By simulating different scenarios, this study aims to estimate the range of potential outcomes and assess the likelihood of achieving the desired down payment amount within the specified time horizon. The use of Monte Carlo simulations allows for a more comprehensive analysis of the potential risks and returns associated with different investment strategies, providing valuable insights for first-time homebuyers.

\subsubsection{Critiques}
One critique of Monte Carlo simulations is their computational intensity, which can be demanding for large-scale applications. Additionally, the accuracy of the results is heavily dependent on the quality of the input data and the assumptions made about the probability distributions of asset returns. Monte Carlo methods also never fully capture the complexities of financial markets, such as changing market conditions and behavioral factors. Moreover, the reliance on historical data may not accurately predict future market behavior, leading to potential misestimations of risk and return.

\newpage
