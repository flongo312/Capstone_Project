\section{Literature Review}

The literature on optimal investment strategies and their applications in real estate and housing markets is extensive. Seminal works by Markowitz (1952) on Modern Portfolio Theory (MPT) and Sharpe (1964) on the Capital Asset Pricing Model (CAPM) provide the foundational theories. Recent studies have expanded on these theories, examining their applications in the context of housing markets (e.g., Smith and Jones, 2020; Lee et al., 2021). This paper aims to fill gaps identified in the literature, particularly the need for tailored investment strategies for first-time homebuyers.

\subsection{Modern Portfolio Theory}
Markowitz's Modern Portfolio Theory (1952) introduced the concept of portfolio optimization, emphasizing the trade-off between risk and return. The theory suggests that investors can achieve optimal portfolios by diversifying their investments across different assets, thereby reducing risk without sacrificing expected returns.

\subsection{Capital Asset Pricing Model}
Sharpe's Capital Asset Pricing Model (1964) extends the notion of risk by introducing systematic and unsystematic risk. The model posits that the expected return on an asset is a function of its sensitivity to market movements (beta), the risk-free rate, and the market risk premium.

\subsection{Recent Studies}
Recent studies have explored the application of these theories in various contexts. Smith and Jones (2020) examined the effectiveness of MPT in constructing retirement portfolios, while Lee et al. (2021) analyzed the impact of CAPM on real estate investment trusts (REITs). These studies highlight the evolving nature of investment strategies and their relevance in contemporary markets.

\newpage
