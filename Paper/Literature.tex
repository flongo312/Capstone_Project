\section{Literature Review}

The literature on optimal investment strategies and their applications in real estate and housing markets is extensive. Seminal works by \citet{markowitz1952portfolio} on Modern Portfolio Theory (MPT) and \citet{sharpe1964capital} on the Capital Asset Pricing Model (CAPM) provide the foundational theories. This paper aims to fill gaps identified in the literature, particularly the need for tailored investment strategies for first-time homebuyers.

\subsection{Foundational Theories}

\subsubsection{Markowitz's Modern Portfolio Theory}
Markowitz's Modern Portfolio Theory \citep{markowitz1952portfolio} introduced the concept of portfolio optimization, emphasizing the trade-off between risk and return. The theory suggests that investors can achieve optimal portfolios by diversifying their investments across different assets, thereby reducing risk without sacrificing expected returns. Markowitz's efficient frontier illustrates the set of optimal portfolios that offer the highest expected return for a given level of risk. Further advancements by \citet{fabozzi2002foundations} have expanded on these foundational concepts, incorporating various market conditions and investor behaviors.

\subsubsection{Sharpe's Capital Asset Pricing Model}
Sharpe's Capital Asset Pricing Model \citep{sharpe1964capital} extends the notion of risk by introducing systematic and unsystematic risk. The model posits that the expected return on an asset is a function of its sensitivity to market movements (beta), the risk-free rate, and the market risk premium. Sharpe's work was pivotal in differentiating between diversifiable (unsystematic) risk and non-diversifiable (systematic) risk, providing a framework for understanding how different types of risk impact asset pricing and returns. Subsequent studies by \citet{fama1973risk} and \citet{black1972capital} have refined the CAPM, addressing its assumptions and limitations.

\subsection{Monte Carlo Simulations}
Monte Carlo simulations have become a critical tool in financial modeling, providing a method for assessing the impact of risk and uncertainty in investment strategies. Boyle \citep{boyle1977options} introduced the Monte Carlo approach to option pricing, and since then, its applications have expanded across various fields of finance. Recent works by \citet{glasserman2004monte} and \citet{kreps2019applications} have further developed these techniques, making them more robust and applicable to diverse financial scenarios.

\subsubsection{Monte Carlo in Portfolio Management}
Monte Carlo simulations are used to model the behavior of investment portfolios under a wide range of possible future scenarios. This method involves generating a large number of random samples from the probability distributions of asset returns and analyzing the resulting portfolio performance. Studies by \citet{glasserman2004monte} and \citet{chan2005monte} have demonstrated how Monte Carlo simulations can provide insights into the risk and return profiles of different investment strategies, helping investors to better understand the potential outcomes and make more informed decisions.

\subsubsection{Monte Carlo in Real Estate Investments}
In the context of real estate, Monte Carlo simulations have been used to evaluate the risk and return of property investments. For example, Brown and Matysiak \citep{brown2000real} employed Monte Carlo methods to assess the uncertainty in real estate portfolio returns, while Kuhle and Alvayay \citep{kuhle2021economic} used simulations to analyze the impact of economic shocks on real estate prices. These applications illustrate the versatility of Monte Carlo simulations in addressing the unique risks associated with real estate investments. Additional studies by \citet{case2003dynamic} and \citet{ling2009real} have also highlighted the effectiveness of Monte Carlo simulations in real estate valuation and risk management.

\subsection{Investment Strategies for First-Time Homebuyers}
Research has shown that first-time homebuyers face unique financial challenges and require tailored investment strategies. Studies by \citet{brueckner2012strategic} and \citet{dietz2003effects} underscore the importance of customized financial planning in achieving homeownership goals. \citet{lee2013savings} explore the role of savings behavior and financial literacy in homeownership attainment. Additionally, \citet{goodman2018accessing} and \citet{mcwilliams2017assessing} discuss the impact of macroeconomic factors and housing policies on first-time homebuyers.

\subsubsection{Risk Management in Homeownership Savings}
Managing risk is crucial for first-time homebuyers aiming to save for a down payment. \citet{baker2010down} examine the effectiveness of different saving strategies under varying economic conditions, while \citet{mason2013risk} provide insights into the role of insurance and financial products in mitigating risks associated with homeownership savings. \citet{rosen2005housing} discuss the implications of housing market volatility on savings strategies and home affordability.

\subsubsection{Role of Financial Technology in Homeownership Savings}
The advent of financial technology has transformed the way individuals save and invest. \citet{phillips2019fintech} explore the impact of fintech solutions on personal savings rates and investment behaviors, highlighting the potential benefits for first-time homebuyers. \citet{gomber2018digital} discuss how digital platforms and robo-advisors can enhance financial planning and investment management for aspiring homeowners.

\subsection{Gaps in the Literature}
Despite the extensive research on investment strategies, there remains a need for tailored approaches that address the specific financial goals and constraints of first-time homebuyers. This paper seeks to fill this gap by developing and evaluating optimal investment strategies that cater to the unique needs of aspiring homeowners across different age cohorts and investment horizons. This includes considering factors such as income levels, regional housing market conditions, and the integration of alternative investment vehicles like REITs and cryptocurrencies.

\newpage
