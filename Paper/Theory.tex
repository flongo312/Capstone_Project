\section{Theoretical Models}

\subsection{Capital Asset Pricing Model (CAPM)}
The Capital Asset Pricing Model (CAPM) is a fundamental tool in finance used to determine the expected return on an asset based on its systematic risk, as measured by beta ($\beta_i$). The CAPM formula is:

\begin{equation}
E(R_i) = R_f + \beta_i(E(R_m) - R_f)
\end{equation}

where:
\begin{itemize}
    \item $E(R_i)$ is the expected return on asset $i$.
    \item $R_f$ is the risk-free rate of return.
    \item $\beta_i$ is the beta of asset $i$, representing its sensitivity to market movements.
    \item $E(R_m)$ is the expected return of the market.
    \item $(E(R_m) - R_f)$ is the market risk premium.
\end{itemize}



\subsubsection{Beta Calculation}
Beta ($\beta_i$) measures the volatility of an asset in relation to the market. It is calculated as:

\begin{equation}
\beta_i = \frac{\text{Cov}(R_i, R_m)}{\sigma^2_m}
\end{equation}

where:
\begin{itemize}
    \item $\text{Cov}(R_i, R_m)$ is the covariance between the return of asset $i$ and the return of the market.
    \item $\sigma^2_m$ is the variance of the market return.
\end{itemize}



\subsubsection{Security Market Line (SML)}
The Security Market Line (SML) is a graphical representation of the CAPM, showcasing the relationship between the expected return of an asset and its systematic risk, as measured by beta ($\beta_i$).

\begin{figure}[!ht]
    \centering
    \includegraphics[width=0.6\textwidth]{../Figures/SML.png}
    \caption{Security Market Line \citep{wikipediaSML}}
    \label{fig:SML}
\subcaption{The Security Market Line (SML) plots the expected return of an asset against its beta, showing that higher systematic risk (beta) correlates with higher expected returns. Correctly priced securities should lie on the SML. The slope represents the market risk premium ($E(R_m) - R_f$), while the intercept is the risk-free rate ($R_f$).}

\end{figure}


\newpage


\subsection{Sharpe Ratio}
The Sharpe Ratio is a measure of risk-adjusted return, providing a way to compare the performance of investments while considering the risk taken. A higher Sharpe Ratio indicates better risk-adjusted performance, meaning the investment provides higher returns for each unit of risk taken. It is calculated as follows:

\begin{equation}
\text{Sharpe Ratio} = \frac{E(R_i) - R_f}{\sigma_i}
\end{equation}

where:
\begin{itemize}
    \item $E(R_i)$ is the expected return of the investment.
    \item $R_f$ is the risk-free rate.
    \item $\sigma_i$ is the standard deviation of the investment's return.
\end{itemize}

\subsection{Composite Score Calculation}
The Composite Score integrates multiple metrics to rank securities. Weights are assigned to each metric to reflect their importance in the ranking process. It is calculated as follows:

\begin{equation}
\text{Composite Score} = w_{\beta} \beta + w_{\text{Sharpe}} \text{Sharpe Ratio} + w_{\text{CAPM}} E(R_i) + w_{\text{Actual}} \text{Actual Returns}
\end{equation}

where:
\begin{itemize}
    \item $w_{\beta}$ is the weight assigned to the beta.
    \item $w_{\text{Sharpe}}$ is the weight assigned to the Sharpe Ratio.
    \item $w_{\text{CAPM}}$ is the weight assigned to the CAPM predicted return.
    \item $w_{\text{Actual}}$ is the weight assigned to the actual returns.
\end{itemize}




\subsection{Modern Portfolio Theory (MPT)}
Modern Portfolio Theory (MPT) provides a robust framework for constructing an optimal portfolio that maximizes expected return for a given level of risk. The expected return $E(R_p)$ of a portfolio is the weighted sum of the expected returns of the individual assets:

\begin{equation}
E(R_p) = \sum_{i=1}^n w_iE(R_i)
\end{equation}

where:
\begin{itemize}
    \item $E(R_p)$ is the expected return of the portfolio.
    \item $w_i$ are the weights of the individual assets in the portfolio.
    \item $E(R_i)$ is the expected return of asset $i$.
\end{itemize}

\subsubsection{Efficient Frontier and Optimal Portfolio}
The efficient frontier is a concept from MPT that represents the set of optimal portfolios offering the highest expected return for a defined level of risk. The process of constructing the efficient frontier involves solving the following optimization problem:

\begin{equation}
\min \sum_{i=1}^n \sum_{j=1}^n w_i w_j \sigma_{ij}
\end{equation}

subject to:

\begin{equation}
\sum_{i=1}^n w_i = 1
\end{equation}

and

\begin{equation}
E(R_p) = \sum_{i=1}^n w_iE(R_i)
\end{equation}

where:
\begin{itemize}
    \item $\sigma_{ij}$ is the covariance between the returns of assets $i$ and $j$.
    \item $w_i$ and $w_j$ are the weights of assets $i$ and $j$ in the portfolio.
\end{itemize}

\begin{figure}[h!]
    \centering
    \includegraphics[width=0.6\textwidth]{../Figures/efficient_frontier.png}
    \caption{Efficient Frontier}
    \label{fig:efficient_frontier}
    \subcaption{This figure shows the efficient frontier, illustrating optimal portfolios that offer the highest expected return for a given level of risk. Portfolios below the efficient frontier are sub-optimal as they do not provide sufficient return for the risk taken.}
\end{figure}


\subsection{Monte Carlo Simulation}
Monte Carlo simulations are utilized to model the uncertainty and variability in investment returns over time. The simulation process involves generating random returns based on historical data and iterating this process to build a distribution of potential outcomes. By running multiple simulations, we can estimate the expected value and variability of the investment portfolio, providing insights into the likelihood of achieving the desired down payment amount within the specified time horizon. The value of an investment at time $i$ is given by:

\begin{equation}
X_i = X_{i-1} \times (1 + r_i)
\end{equation}

where:
\begin{itemize}
    \item $X_i$ is the investment value at time $i$.
    \item $X_{i-1}$ is the investment value at time $i-1$.
    \item $r_i$ is the return for period $i$.
\end{itemize}


\begin{figure}[h!]
    \centering
    \includegraphics[width=0.6\textwidth]{../Figures/investment_simulation_process.png}
    \caption{Investment Simulation Process for Monte Carlo Analysis}
    \label{fig:investment_simulation}
    \subcaption{The diagram illustrates the generation of random returns used to project the future value of an investment over multiple iterations, creating a range of possible outcomes to understand potential risks and returns.}
\end{figure}


\newpage
