\section{Results}
The empirical analysis revealed significant findings. Table \ref{performance-comparison} compares the performance of our optimal portfolios against the S\&P 500 over different horizons.

\begin{table}[h!]
\centering
\begin{tabular}{lccc}
\hline
Horizon & Optimal Portfolio Return & S\&P 500 Return & Difference \\
\hline
5 Years & 441.66\% & 375.45\% & 66.21\% \\
7.5 Years & 454.96\% & 389.22\% & 65.74\% \\
10 Years & 314.81\% & 298.54\% & 16.27\% \\
\hline
\end{tabular}
\caption{Performance Comparison of Optimal Portfolios and S\&P 500}
\label{performance-comparison}
\end{table}

\subsection{Scenario Analysis}
A scenario analysis was conducted to assess how different economic conditions (e.g., recession, boom, interest rate changes) impact the optimal investment strategies. This analysis provides practical insights into the resilience of the strategies under various economic scenarios.

\subsection{Sensitivity Analysis}
A sensitivity analysis was performed to understand how changes in key assumptions (e.g., risk-free rate, market return) affect the outcomes of the investment strategies. This analysis provides insights into the robustness of our findings under different economic conditions.

\subsection{Visual Explanations}
Figures \ref{fig:security-count}, \ref{fig:optimal-portfolio-5-years}, \ref{fig:top-assets-10-years}, \ref{fig:top-assets-7-5-years}, \ref{fig:top-assets-5-years}, and \ref{fig:cumulative-returns} provide visual representations of the distribution of security types, optimal portfolios, top assets by composite score, and cumulative returns by security type. Each chart includes detailed explanations to help readers understand the implications of the visual data.

