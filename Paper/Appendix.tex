\appendix
\section{Data Appendix}

\subsection{Summary of Data Sources}
The financial data utilized in this study were obtained from Yahoo Finance, which provides comprehensive information on a range of securities including stocks, mutual funds, and ETFs.

\subsection{Data Cleaning and Processing Details}
Detailed steps taken to clean and process the data include:
\begin{itemize}
    \item Adjusting for corporate actions like stock splits and dividends to maintain consistency in price data.
    \item Interpolating missing values to handle gaps in the dataset.
    \item Detecting and handling outliers to prevent skewed results.
\end{itemize}

\subsection{Python Scripts for Data Processing}
Several Python scripts were developed to automate data collection and processing:
\begin{itemize}
    \item \texttt{yfinance\_data.py}: Collects and processes financial data from Yahoo Finance.
    \item \texttt{hindsight\_data.py}: Handles historical financial data for hindsight analysis.
    \item \texttt{init\_filtering.py}: Filters the initial dataset to remove anomalies and irrelevant data points.
\end{itemize}

\subsection{Variable Definitions}
The key variables used in the analysis are defined as follows:
\begin{itemize}
    \item \textbf{Open}: The price at the beginning of the trading day.
    \item \textbf{High}: The highest price during the trading day.
    \item \textbf{Low}: The lowest price during the trading day.
    \item \textbf{Close}: The price at the end of the trading day.
    \item \textbf{Adj Close}: The closing price adjusted for dividends, stock splits, etc.
    \item \textbf{Volume}: The number of shares traded during the trading day.
    \item \textbf{Type}: The type of security (e.g., stock, ETF).
    \item \textbf{Ticker}	Beta	CAPM Predicted Return	Sharpe Ratio	Actual Returns	Type	Composite Score 5 Years	Rank
\end{itemize}

\subsection{Summary Statistics}
Summary statistics for the dataset are presented in Table \ref{tab:summary-stats}. These statistics were generated using the \texttt{../Data/summary\_stats.csv} file produced by the Python scripts.

\begin{table}[h!]
\centering
\scriptsize
\begin{tabular}{lccccccc}
\hline
Statistic & \begin{tabular}[c]{@{}c@{}}Mean Final \\ Portfolio Value (\$)\end{tabular} & \begin{tabular}[c]{@{}c@{}}Median Final \\ Portfolio Value (\$)\end{tabular} & \begin{tabular}[c]{@{}c@{}}10th Percentile Final \\ Portfolio Value (\$)\end{tabular} & \begin{tabular}[c]{@{}c@{}}90th Percentile Final \\ Portfolio Value (\$)\end{tabular} & \begin{tabular}[c]{@{}c@{}}Total Percentage \\ Yield (\%)\end{tabular} & \begin{tabular}[c]{@{}c@{}}Annual Percentage \\ Yield (\%)\end{tabular} \\
\hline
10-Year Horizon & 283094.44 & 265864.65 & 171232.59 & 420935.51 & 2730.94 & 39.70 \\
7.5-Year Horizon & 180745.35 & 173433.88 & 116972.91 & 256760.32 & 1707.45 & 47.10 \\
5-Year Horizon & 101559.42 & 98327.18 & 67062.20 & 139233.79 & 915.59 & 58.98 \\
\hline
\end{tabular}
\caption{Summary Statistics of the Dataset}
\label{tab:summary-stats}
\end{table}




\newpage
