\appendix
\section{Data Appendix}

\subsection{Summary of Data Sources}
The financial data utilized in this study were obtained from Yahoo Finance, which provides comprehensive information on a range of securities including stocks, mutual funds, and ETFs \citep{yfinance}.

\subsection{Variable Definitions}
The key variables used in the analysis are defined as follows:
\begin{itemize}
    \item \textbf{Open}: The price at the beginning of the trading day.
    \item \textbf{High}: The highest price during the trading day.
    \item \textbf{Low}: The lowest price during the trading day.
    \item \textbf{Close}: The price at the end of the trading day.
    \item \textbf{Adj Close}: The closing price adjusted for dividends, stock splits, etc.
    \item \textbf{Volume}: The number of shares traded during the trading day.
    \item \textbf{Type}: The type of security (e.g., stock, ETF).
    \item \textbf{Ticker}: The unique symbol assigned to each security for trading purposes.
    \item \textbf{Beta}: A measure of a security's volatility in relation to the overall market. A beta greater than 1 indicates that the security is more volatile than the market, while a beta less than 1 indicates that it is less volatile.
    \item \textbf{CAPM Predicted Return}: The expected return of a security as predicted by the Capital Asset Pricing Model, which takes into account the risk-free rate, the security's beta, and the expected market return.
    \item \textbf{Sharpe Ratio}: A measure of risk-adjusted return, calculated by subtracting the risk-free rate from the security's return and dividing by the standard deviation of the security's return. A higher Sharpe Ratio indicates better risk-adjusted performance.
    \item \textbf{Actual Returns}: The realized return on a security over a specified period, including price appreciation and dividends.
    \item \textbf{Composite Score}: A combined score that integrates various metrics such as Beta, CAPM Predicted Return, Actual Returns, and Sharpe Ratio to evaluate the overall attractiveness of a security. The composite score is used to rank securities.
    \item \textbf{Rank}: The position of a security in the list based on its composite score, with a lower rank indicating a more attractive investment.
\end{itemize}

\newpage
