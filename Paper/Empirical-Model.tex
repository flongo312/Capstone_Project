\section{Empirical Specification}

\subsection{Model Implementation}
The empirical analysis involves a systematic approach to evaluating and optimizing investment strategies using historical data. The data is cleaned and adjusted for corporate actions like stock splits and dividends using the \texttt{yfinance\_data.py} script to ensure consistency, resulting in the creation of a file named \texttt{yfinance\_data.csv}.

\subsubsection{CAPM and Sharpe Ratio Calculation}
Each security's risk and return profile is assessed using the Capital Asset Pricing Model (CAPM) and Sharpe Ratio. This analysis is performed using the \texttt{init\_filtering.py} script.
\begin{enumerate}
    \item Calculating the average return of the market index (e.g., S\&P 500).
    \item Determining the risk-free rate (e.g., 10-year U.S. Treasury bonds yield).
    \item Computing the beta (\(\beta_i\)) of each security:
    \begin{equation}
        \beta_i = \frac{\text{Cov}(R_i, R_m)}{\text{Var}(R_m)}
    \end{equation}
    \item Estimating the expected return using the CAPM formula:
    \begin{equation}
        E(R_i) = R_f + \beta_i (E(R_m) - R_f)
    \end{equation}
    \item Calculating the Sharpe Ratio:
    \begin{equation}
        \text{Sharpe Ratio} = \frac{E(R_i) - R_f}{\sigma_i}
    \end{equation}
\end{enumerate}

\subsubsection{Composite Score Calculation}
Securities are ranked based on a composite score that integrates multiple metrics:
\begin{equation}
    \text{Composite Score} = w_{\beta} \beta + w_{\text{Sharpe}} \text{Sharpe Ratio} + w_{\text{CAPM}} E(R_i) + w_{\text{Actual}} \text{Actual Returns}
\end{equation}
Weights are assigned to each metric to reflect their importance in the ranking process. A file is then produced for each time horizon with the ranked securities in order, named \texttt{top\_assets\_composite\_score.csv}.

\subsubsection{Modern Portfolio Theory (MPT) Application}
Using the ranked securities, portfolios are optimized for different investment horizons (5, 7.5, and 10 years) using Modern Portfolio Theory (MPT). The optimization involves solving the following quadratic programming problem:
\begin{align}
    \text{Maximize } & \quad \mathbf{w}^T \mathbf{\mu} - \frac{\lambda}{2} \mathbf{w}^T \mathbf{\Sigma} \mathbf{w} \\
    \text{subject to} & \quad \sum_{i} w_i = 1 \\
    & \quad w_i \geq 0 \quad \forall i
\end{align}
where \(\mathbf{w}\) is the vector of asset weights, \(\mathbf{\mu}\) is the vector of expected returns, \(\mathbf{\Sigma}\) is the covariance matrix of returns, and \(\lambda\) is the risk aversion parameter. This process is implemented using the \texttt{mpt.py} script to produce a data file for each time horizon named \texttt{optimal\_weights.csv}.

\subsubsection{Performance Comparison}
The optimized portfolios are compared against actual historical performance using the hindsight data created by \texttt{hindsight\_data.py}. The comparison includes calculating the cumulative returns of the portfolios and benchmarking them against the S\&P 500 index. This step is performed using the \texttt{hindsight.py} script to create a data file named \texttt{hindsight\_data.csv}.

\subsubsection{Monte Carlo Simulation for Future Forecasting}
Monte Carlo simulations are conducted to forecast the performance of the optimized portfolios. This comprehensive simulation is implemented using the \texttt{mcs.py} script, providing insights into the potential future performance of the portfolios.
\begin{enumerate}
    \item Defining initial investment amounts and annual contributions.
    \item Generating random returns based on historical distributions.
    \item Calculating portfolio values at each time step:
    \begin{equation}
        V_t = V_{t-1} \times (1 + R_t) + C
    \end{equation}
    \item Running multiple iterations to build a probability distribution of outcomes.
    \item Applying economic shocks to simulate real-world scenarios:
    \begin{equation}
        V_t = V_t \times (1 + \text{Shock Intensity})
    \end{equation}
\end{enumerate}

\newpage
