\section{Introduction and Motivation}
The escalating housing costs in contemporary real estate markets have created significant barriers for first-time homebuyers. This demographic often faces the daunting task of accumulating substantial down payments amidst economic volatility and uncertain income trajectories. This research addresses this critical issue by developing and evaluating optimal investment strategies tailored to help diverse age groups achieve their homeownership goals within 5, 7.5, and 10-year horizons.

\subsection{Literature Review}
The literature on optimal investment strategies and their applications in real estate and housing markets is extensive. Seminal works by Markowitz (1952) on Modern Portfolio Theory (MPT) and Sharpe (1964) on the Capital Asset Pricing Model (CAPM) provide the foundational theories. Recent studies have expanded on these theories, examining their applications in the context of housing markets (e.g., Smith and Jones, 2020; Lee et al., 2021). This paper aims to fill gaps identified in the literature, particularly the need for tailored investment strategies for first-time homebuyers.

\subsection{Theoretical Frameworks}
This study leverages several key financial theories. The CAPM is employed to determine the expected return on an asset based on its systematic risk. Modern Portfolio Theory (MPT) provides a robust framework for constructing an optimal portfolio that maximizes expected return for a given level of risk. Additionally, Monte Carlo simulations are used to model the uncertainty and variability in investment returns over time.

\subsection{Research Question}
The central research question guiding this investigation is: What are the most effective investment strategies for different age groups to accumulate housing down payments over periods of 5, 7.5, and 10 years?

\subsection{Objective}
The primary objective of this study is to identify, analyze, and optimize investment strategies that can effectively assist first-time homebuyers in saving for their down payments. By leveraging advanced financial theories and empirical methodologies, this research aims to provide actionable insights that balance risk and return, offering practical solutions for prospective homeowners.

\subsection{Motivation}
The motivation for this research stems from the pressing need to address the challenges posed by rising housing costs and economic instability. As homeownership becomes increasingly out of reach for many, particularly younger individuals, it is imperative to develop strategies that can mitigate these barriers. By providing evidence-based investment strategies, this study aims to empower individuals with the tools needed to navigate the complexities of financial planning for homeownership.
