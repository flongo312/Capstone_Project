\documentclass{beamer}
\usetheme{Boadilla}
\usecolortheme{dolphin}
\usefonttheme{serif}
\setbeamertemplate{navigation symbols}{}
\setbeamertemplate{caption}[numbered]
\usepackage{graphicx}
\usepackage{amsmath}
\usepackage{hyperref}
\usepackage{booktabs}
\usepackage{multicol}
\usepackage{pgfplots}
\pgfplotsset{compat=1.18}

\title{An Economic Analysis of Optimal Investment Strategies for Accumulating Housing Down Payments}
\author{Frank Paul Longo II}
\date{\today}

\begin{document}

\begin{frame}
    \titlepage
\end{frame}

\begin{frame}{Agenda}
    \tableofcontents
\end{frame}

\section{Introduction}
\begin{frame}{Introduction}
  \begin{itemize}
    \item \textbf{Objective:} To develop tailored investment strategies for different age groups of first-time homebuyers to accumulate funds for down payments.
    \item \textbf{Motivation:} Rising housing costs present significant challenges for many first-time homebuyers, particularly younger individuals. This research aims to identify optimal investment strategies to help people in various age groups save effectively for a down payment, thereby accelerating their path to homeownership.
  \end{itemize}
\end{frame}

\section{Why First-time Homebuyers Need This Help}
\begin{frame}{Typical First-time Homebuyer Profile}
  \begin{itemize}
    \item \textbf{Average Age:} 35 years (as of 2023)
    \item \textbf{Median Income:} \$95,900 (as of 2023)
    \item \textbf{Marital Status:}
    \begin{itemize}
        \item 59\% Married Couples
        \item 19\% Single Females
        \item 10\% Single Males
        \item 9\% Unmarried Couples
    \end{itemize}
    \item \textbf{Challenges:}
    \begin{itemize}
        \item High student loan debt (national average: \$37,172 in 2023)
        \item Rising housing costs exceeding income growth
        \item Difficulty saving for a down payment
    \end{itemize}
    \item \textbf{Average Home Cost:} \$348,000 (2022 data)
    \item \textbf{Down Payment:}
    \begin{itemize}
        \item \textbf{Average Saved:} \$8,220 (6\% of average home price in 2023)
    \end{itemize}
  \end{itemize}
\end{frame}

\section{Data Sources and Analysis}
\begin{frame}{Data Sources Utilized}
    \begin{itemize}
        \item \textbf{Yahoo Finance (YFinance):}
        \begin{itemize}
            \item Comprehensive financial data on stocks, cryptocurrency, mutual funds, and ETFs.
        \end{itemize}
    \end{itemize}
\end{frame}

\begin{frame}{Overview of Financial Data from Yahoo Finance}
    \begin{itemize}
        \item \textbf{Date Range:} 9/7/2014 to present (daily frequency)
        \item \textbf{Data Fields:}
        \begin{itemize}
            \item \textbf{Open:} Price at the beginning of the trading day
            \item \textbf{High:} Peak price during the trading day
            \item \textbf{Low:} Lowest price during the trading day
            \item \textbf{Close:} Price at the end of the trading day
            \item \textbf{Adj Close:} Closing price adjusted for dividends, stock splits, etc.
            \item \textbf{Volume:} Number of shares traded during a single trading day
            \item \textbf{Type:} Security type
        \end{itemize}
    \end{itemize}
\end{frame}

\section{Financial Literacy: Essential Concepts}
\begin{frame}{Essential Financial Concepts}
    \begin{itemize}
        \item \textbf{Stocks:}
        \begin{itemize}
            \item Equity investments representing ownership in a company.
            \item \textbf{Example:} Purchasing shares of Apple Inc. (AAPL).
        \end{itemize}
        \item \textbf{Cryptocurrency:}
        \begin{itemize}
            \item Digital or virtual currencies that use cryptography for security.
            \item \textbf{Example:} Bitcoin (BTC) and Ethereum (ETH).
        \end{itemize}
    \end{itemize}
\end{frame}

\begin{frame}{Essential Financial Concepts (cont.)}
    \begin{itemize}
        \item \textbf{Mutual Funds:}
        \begin{itemize}
            \item Investment vehicles that pool money from many investors to purchase a diversified portfolio of stocks, bonds, or other securities.
            \item Managed by professional fund managers.
            \item \textbf{Example:} Vanguard 500 Index Fund.
        \end{itemize}
        \item \textbf{ETFs (Exchange-Traded Funds):}
        \begin{itemize}
            \item Similar to mutual funds but traded on stock exchanges like individual stocks.
            \item Provide diversification and are typically more cost-effective.
            \item \textbf{Example:} SPDR S\&P 500 ETF (SPY).
        \end{itemize}
    \end{itemize}
\end{frame}

\section{CAPM: Capital Asset Pricing Model}
\begin{frame}{Capital Asset Pricing Model (CAPM) - Key Assumptions}
  \begin{block}{Key Assumptions}
    \begin{itemize}
      \item Investors hold diversified portfolios.
      \item Markets are efficient, and all investors have access to the same information.
      \item There are no taxes or transaction costs.
      \item The risk-free rate is constant.
    \end{itemize}
  \end{block}
\end{frame}

\begin{frame}{Step 1: Identify Risk-Free Rate}
  \begin{block}{Risk-Free Rate (\(R_f\))}
    The risk-free rate is the return on an investment with zero risk, typically represented by government bonds.
    \begin{equation*}
      R_f = \text{Risk-free rate}
    \end{equation*}
    Example:
    \begin{itemize}
      \item Assume the risk-free rate is 3\%.
    \end{itemize}
  \end{block}
\end{frame}

\begin{frame}{Step 2: Determine Market Return}
  \begin{block}{Market Return (\(E(R_m)\))}
    The expected return of the market is the average return of the market portfolio, which includes all investable assets.
    \begin{equation*}
      E(R_m) = \text{Expected return of the market}
    \end{equation*}
    Example:
    \begin{itemize}
      \item Assume the expected market return is 8\%.
    \end{itemize}
  \end{block}
\end{frame}

\begin{frame}{Step 3: Calculate Asset Beta}
  \begin{block}{Beta (\(\beta\))}
    Beta is a measure of an asset's volatility relative to the overall market. It indicates the asset's systematic risk.
    \begin{equation*}
      \beta_i = \frac{\text{Cov}(R_i, R_m)}{\sigma_m^2}
    \end{equation*}
    where:
    \begin{itemize}
      \item \( \text{Cov}(R_i, R_m) \) = Covariance of asset \( i \) with the market
      \item \( \sigma_m^2 \) = Variance of the market returns
    \end{itemize}
    Example Calculation:
    \begin{equation*}
      \beta_i = \frac{0.015}{0.02} = 0.75
    \end{equation*}
  \end{block}
\end{frame}

\begin{frame}{Step 4: Calculate Expected Return Using CAPM Formula}
  \begin{block}{CAPM Formula}
    The CAPM formula calculates the expected return of an asset based on its systematic risk (beta).
    \begin{equation*}
      E(R_i) = R_f + \beta_i (E(R_m) - R_f)
    \end{equation*}
    where:
    \begin{itemize}
      \item \( E(R_i) \) = Expected return of asset \( i \)
      \item \( R_f \) = Risk-free rate
      \item \( \beta_i \) = Beta of asset \( i \)
      \item \( E(R_m) \) = Expected return of the market
    \end{itemize}
    Example Calculation:
    \begin{equation*}
      E(R_i) = 3\% + 0.75 (8\% - 3\%) = 6.75\%
    \end{equation*}
  \end{block}
\end{frame}

\begin{frame}{Step 5: Plot the Security Market Line (SML)}
  \begin{block}{Security Market Line (SML)}
    The Security Market Line (SML) is a graphical representation of the CAPM. It plots the expected return of an asset against its beta.
    \begin{equation*}
      E(R_i) = R_f + \beta_i (E(R_m) - R_f)
    \end{equation*}
    \begin{itemize}
      \item The y-intercept represents the risk-free rate (\(R_f\)).
      \item The slope represents the market risk premium (\(E(R_m) - R_f\)).
    \end{itemize}
  \end{block}
  \begin{block}{Explanation}
    The SML illustrates the trade-off between risk and return for efficient portfolios. Assets plotted above the SML are considered undervalued, while those below the SML are considered overvalued.
  \end{block}
\end{frame}

\begin{frame}{Step 6: Interpret the Results}
  \begin{block}{Interpreting the CAPM}
    Use the CAPM results to make informed investment decisions.
    \begin{itemize}
      \item Compare the expected return calculated by CAPM with the actual return of the asset.
      \item Determine if the asset is fairly valued, undervalued, or overvalued based on its position relative to the SML.
    \end{itemize}
    Example:
    \begin{itemize}
      \item If the actual return of an asset is 8\% but the CAPM expected return is 6.75\%, the asset may be undervalued.
    \end{itemize}
  \end{block}
\end{frame}

\section{Modern Portfolio Theory (MPT)}
\begin{frame}{Modern Portfolio Theory (MPT) - Overview}
  \begin{block}{Overview}
    Modern Portfolio Theory (MPT), introduced by Harry Markowitz, is a framework for constructing a portfolio of assets such that the expected return is maximized for a given level of risk. It emphasizes diversification to reduce risk.
  \end{block}
\end{frame}

\begin{frame}{Modern Portfolio Theory (MPT) - Key Assumptions}
  \begin{block}{Key Assumptions}
    \begin{itemize}
      \item Investors are rational and risk-averse.
      \item Markets are efficient, and all investors have access to the same information.
      \item Asset returns are normally distributed.
      \item There are no transaction costs or taxes.
    \end{itemize}
  \end{block}
\end{frame}

\begin{frame}{Step 1: Define Assets and Expected Returns}
  \begin{block}{Expected Returns}
    Identify the assets you want to include in the portfolio and estimate their expected returns (\(E(R_i)\)). This involves analyzing historical data, considering economic conditions, and using financial models.
    \begin{equation*}
      E(R_i) = \text{Expected return of asset } i
    \end{equation*}
    Example: 
    \begin{itemize}
      \item Asset A: \(E(R_A) = 10\%\)
      \item Asset B: \(E(R_B) = 15\%\)
    \end{itemize}
  \end{block}
\end{frame}

\begin{frame}{Step 2: Determine Asset Weights}
  \begin{block}{Weights}
    Decide on the proportion (\(w_i\)) of the total investment to allocate to each asset. The sum of the weights should equal 1.
    \begin{equation*}
      w_i = \text{Weight of asset } i
    \end{equation*}
    Example:
    \begin{itemize}
      \item Weight of Asset A: \(w_A = 60\%\)
      \item Weight of Asset B: \(w_B = 40\%\)
    \end{itemize}
  \end{block}
\end{frame}

\begin{frame}{Step 3: Calculate Portfolio's Expected Return}
  \begin{block}{Expected Return of Portfolio}
    The expected return of the portfolio (\(E(R_p)\)) is the weighted sum of the expected returns of the individual assets.
    \begin{equation*}
      E(R_p) = \sum_{i=1}^{n} w_i E(R_i)
    \end{equation*}
    where:
    \begin{itemize}
      \item \( E(R_p) \) = Expected return of the portfolio
      \item \( w_i \) = Weight of asset \( i \) in the portfolio
      \item \( E(R_i) \) = Expected return of asset \( i \)
    \end{itemize}
    Example Calculation:
    \begin{equation*}
      E(R_p) = (0.60 \times 0.10) + (0.40 \times 0.15) = 0.12 \text{ or 12\%}
    \end{equation*}
  \end{block}
\end{frame}

\begin{frame}{Step 4: Calculate Covariances Between Assets}
  \begin{block}{Covariance}
    Covariance measures how two assets move together. A positive covariance means that the assets tend to move in the same direction, while a negative covariance means they move in opposite directions.
    \begin{equation*}
      \sigma_{ij} = \text{Cov}(R_i, R_j) = \mathbb{E}[(R_i - \mathbb{E}[R_i])(R_j - \mathbb{E}[R_j])]
    \end{equation*}
    Example:
    \begin{itemize}
      \item Covariance between Asset A and Asset B: \(\sigma_{AB} = 0.02\)
    \end{itemize}
  \end{block}
\end{frame}

\begin{frame}{Step 5: Calculate Portfolio's Variance (Risk)}
  \begin{block}{Variance of Portfolio}
    The variance (\(\sigma_p^2\)) of the portfolio's return is determined by the variances of the individual assets and the covariances between them.
    \begin{equation*}
      \sigma_p^2 = \sum_{i=1}^{n} \sum_{j=1}^{n} w_i w_j \sigma_{ij}
    \end{equation*}
    where:
    \begin{itemize}
      \item \( \sigma_p^2 \) = Variance of the portfolio's return
      \item \( w_i \) = Weight of asset \( i \) in the portfolio
      \item \( \sigma_{ij} \) = Covariance between asset \( i \) and asset \( j \)
    \end{itemize}
    Example Calculation:
    \begin{equation*}
      \sigma_p^2 = (0.60)^2 \times 0.04 + (0.40)^2 \times 0.09 + 2 \times 0.60 \times 0.40 \times 0.02 = 0.0384
    \end{equation*}
  \end{block}
\end{frame}

\begin{frame}{Step 6: Calculate Correlation Between Assets}
  \begin{block}{Correlation}
    Correlation is a standardized measure of covariance that ranges from -1 to 1.
    \begin{equation*}
      \rho_{ij} = \frac{\sigma_{ij}}{\sigma_i \sigma_j}
    \end{equation*}
    where:
    \begin{itemize}
      \item \( \rho_{ij} \) = Correlation coefficient between asset \( i \) and asset \( j \)
      \item \( \sigma_{ij} \) = Covariance between asset \( i \) and asset \( j \)
      \item \( \sigma_i \) = Standard deviation of asset \( i \)
      \item \( \sigma_j \) = Standard deviation of asset \( j \)
    \end{itemize}
    Example:
    \begin{itemize}
      \item Correlation between Asset A and Asset B: \(\rho_{AB} = \frac{0.02}{\sqrt{0.04} \cdot \sqrt{0.09}} = 0.333\)
    \end{itemize}
  \end{block}
\end{frame}

\begin{frame}{Step 7: Optimize the Portfolio}
  \begin{block}{Optimization}
    Adjust the weights of the assets to maximize the portfolio's expected return for a given level of risk or to minimize risk for a given level of expected return. This is done by solving the optimization problem:
    \begin{equation*}
      \min \sigma_p^2 = \sum_{i=1}^{n} \sum_{j=1}^{n} w_i w_j \sigma_{ij}
    \end{equation*}
    subject to:
    \begin{equation*}
      \sum_{i=1}^{n} w_i = 1 \quad \text{and} \quad E(R_p) = \sum_{i=1}^{n} w_i E(R_i)
    \end{equation*}
    Example:
    \begin{itemize}
      \item Adjust weights \(w_i\) to find the optimal portfolio that minimizes risk for a given expected return.
    \end{itemize}
  \end{block}
\end{frame}

\begin{frame}{Step 8: Construct the Efficient Frontier}
  \begin{block}{Efficient Frontier}
    The efficient frontier represents the set of optimal portfolios that offer the highest expected return for a given level of risk. By solving the optimization problem repeatedly for different levels of expected return, you can plot the efficient frontier.
    \begin{itemize}
      \item Plot the portfolios on a graph with risk (standard deviation) on the x-axis and expected return on the y-axis.
    \end{itemize}
    Example:
    \begin{itemize}
      \item The curve shows the optimal portfolios, with points above the curve being unachievable and points below the curve being inefficient.
    \end{itemize}
  \end{block}
\end{frame}

\section{Monte Carlo Simulation}
\begin{frame}{Transition from MPT to Monte Carlo Simulation}
  \begin{block}{From MPT to Monte Carlo Simulation}
    While the MPT helps in understanding.....(insert info here), Monte Carlo Simulation provides a robust method for modeling the probability of different outcomes by incorporating randomness and uncertainty.
    \begin{itemize}
      \item Monte Carlo Simulation introduces stochastic processes to evaluate the range of possible outcomes, providing a more comprehensive risk assessment.
    \end{itemize}
  \end{block}
\end{frame}

\begin{frame}{Monte Carlo Simulation - Formula}
  \begin{block}{Formula}
    \begin{equation*}
      \text{E}(X) = \frac{1}{N} \sum_{i=1}^{N} f(X_i)
    \end{equation*}
    where:
    \begin{itemize}
      \item \( \text{E}(X) \) = Expected value of the outcome
      \item \( N \) = Number of simulations
      \item \( f(X_i) \) = Function of the simulated variable \( X_i \)
    \end{itemize}
  \end{block}
  \begin{block}{Explanation}
    Monte Carlo Simulation uses repeated random sampling to obtain numerical results. The function \( f(X_i) \) represents the process being simulated. \( \text{E}(X) \) is the average result of all simulations.
  \end{block}
\end{frame}

\section{Conclusion}
\begin{frame}{Conclusion}
    \begin{itemize}
        \item \textbf{Future Directions:} Explore optimal investment strategies tailored for first-time homebuyers to accumulate housing down payments, incorporating modern financial theories and data-driven insights.
        \item Effective strategies include diversified portfolios, the application of MPT, and lifecycle investing to navigate the unique financial challenges faced by first-time homebuyers.
        \item Ongoing research will focus on refining these strategies and exploring their practical applications to further assist first-time homebuyers in achieving their homeownership goals.
    \end{itemize}
\end{frame}

\section{Q\&A}
\begin{frame}{Q\&A}
    \begin{itemize}
        \item \textbf{Clarifications:} Please feel free to ask for any clarifications or additional details regarding the presented research and findings.
    \end{itemize}
\end{frame}

\section{References}
\begin{frame}{References (1/2)}
    \begin{itemize}
        \item Yahoo Finance. (n.d.). Retrieved from \url{https://finance.yahoo.com/}
        \item Robinhood. (n.d.). Retrieved from \url{https://robinhood.com/}
        \item Coinbase. (n.d.). Retrieved from \url{https://www.coinbase.com/}
        \item Investment Company Institute. (n.d.). Retrieved from \url{https://www.ici.org/}
        \item Sharpe, W. F. (1966). Mutual Fund Performance. \textit{Journal of Business, 39}(1), 119-138.
        \item Markowitz, H. (1952). Portfolio Selection. \textit{Journal of Finance, 7}(1), 77-91.
        \item Merton, R. C. (1973). Theory of Rational Option Pricing. \textit{Bell Journal of Economics and Management Science, 4}(1), 141-183.
        \item Ulam, S. (1987). Adventures of a Mathematician. \textit{Charles Scribner's Sons}.
    \end{itemize}
\end{frame}

\begin{frame}{References (2/2)}
    \begin{itemize}
        \item National Association of Realtors. (2023). 2023 Home Buyer and Seller Generational Trends. Retrieved from \url{https://www.nar.realtor/research-and-statistics/research-reports/home-buyer-and-seller-generational-trends}
        \item Federal Reserve. (2023). Report on the Economic Well-Being of U.S. Households in 2023. Retrieved from \url{https://www.federalreserve.gov/publications/2023-economic-well-being-of-us-households-in-2023.htm}
    \end{itemize}
\end{frame}

\end{document}
