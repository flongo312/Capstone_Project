\documentclass{beamer}
\usetheme{Madrid}
\usecolortheme{seahorse}
\usefonttheme{professionalfonts}
\setbeamertemplate{navigation symbols}{}

\usepackage{graphicx}
\usepackage{amsmath}
\usepackage{hyperref}
\usepackage{booktabs}
\usepackage{multicol}
\usepackage{pgfplots}
\pgfplotsset{compat=1.18}

\title{An Economic Analysis of Optimal Investment Strategies for Accumulating Housing Down Payments Among Generation Z in the United States}
\author{Frank Paul Longo II}
\date{\today}

\begin{document}

\frame{\titlepage}

\begin{frame}{Roadmap}
    \tableofcontents
\end{frame}

\section{Introduction}
\begin{frame}{Introduction}
    \begin{itemize}
        \item \textbf{Objective:} Identify and analyze optimal investment strategies tailored for Generation Z to accumulate funds for housing down payments.
        \item \textbf{Focus:} Provide actionable financial strategies addressing unique economic challenges and opportunities faced by Generation Z.
        \item \textbf{Motivation:} 
        \begin{itemize}
            \item Initially intended to analyze options trading strategies, such as writing (selling) covered calls.
            \item Found that options trading has a high barrier to entry for everyday retail investors, which includes Generation Z.
            \item Shifted focus to more accessible investment strategies that align with the financial capabilities and needs of Generation Z.
        \end{itemize}
    \end{itemize}
\end{frame}

\section{Significance of the Study}
\begin{frame}{Significance of the Study}
    \begin{itemize}
        \item \textbf{Achieving Homeownership:} A critical financial milestone contributing significantly to long-term economic stability and wealth accumulation.
        \item \textbf{Addressing Challenges:} Generation Z encounters unprecedented financial hurdles, including high student debt, escalating living costs, and persistent inflation.
        \item \textbf{Strategic Goal:} Develop robust investment strategies to help Generation Z overcome these challenges and realize their homeownership aspirations.
    \end{itemize}
\end{frame}

\section{Why Generation Z Needs This Help}
\begin{frame}{Why Generation Z Needs This Help}
    \begin{itemize}
        \item \textbf{High Student Debt:} Average student loan debt of approximately \$30,000.
        \item \textbf{Escalating Living Costs:} Rising cost of living, including rent, utilities, and daily expenses.
        \item \textbf{Inflation:} Erodes the purchasing power of savings.
        \item \textbf{Economic Uncertainty:} Job market volatility and economic recessions.
        \item \textbf{Financial Literacy:} Many lack the financial literacy required to navigate investment options effectively.
    \end{itemize}
\end{frame}

\section{Statistics on Generation Z}
\begin{frame}{Statistics on Generation Z}
    \begin{columns}
        \column{0.5\textwidth}
        \begin{itemize}
            \item \textbf{Population Size:} Over 68 million individuals in the United States.
            \item \textbf{Education:} The most educated generation, with 59\% pursuing higher education.
            \item \textbf{Employment:} About 40\% work part-time or are self-employed.
        \end{itemize}
        \column{0.5\textwidth}
        \begin{itemize}
            \item \textbf{Savings Habits:} Approximately 72\% have started saving, but only 29\% feel confident about their financial future.
            \item \textbf{Investment Interest:} Around 40\% are interested in learning about investing, yet only 17\% actively invest in the stock market.
        \end{itemize}
    \end{columns}
\end{frame}

\section{Profile of Generation Z}
\begin{frame}{Profile of Generation Z}
        \begin{itemize}
            \item \textbf{Digital Natives:} Born between 1997 and 2012.
            \item \textbf{Characteristics:} Digitally savvy, socially conscious, value financial security.
            \item \textbf{Financial Behaviors:} Cautious financial behavior, strong preference for financial education and planning tools.
        \end{itemize}
\end{frame}

\section{Data Sources Utilized}
\begin{frame}{Data Sources Utilized}
        \begin{itemize}
            \item \textbf{Yahoo Finance (YFinance):}
            \begin{itemize}
                \item Comprehensive financial data on stocks, bonds, and other investment vehicles.
            \end{itemize}
            \item \textbf{Federal Reserve Economic Data (FRED):}
            \begin{itemize}
                \item Essential economic indicators such as inflation rates, interest rates, and unemployment statistics.
            \end{itemize}
        \end{itemize}
\end{frame}

\begin{frame}{Overview of Economic Indicators Data from FRED}
    \begin{columns}
        \column{0.5\textwidth}
        \begin{itemize}
            \item \textbf{Consumer Price Index (CPI)}
            \item \textbf{Producer Price Index (PPI)}
            \item \textbf{Unemployment Rate}
            \item \textbf{Nonfarm Payroll Employment}
            \item \textbf{Industrial Production Index}
        \end{itemize}
        \column{0.5\textwidth}
        \begin{itemize}
            \item \textbf{Retail Sales}
            \item \textbf{Housing Starts and Building Permits}
            \item \textbf{Personal Income and PCE}
            \item \textbf{Capacity Utilization}
            \item \textbf{Durable Goods Orders}
        \end{itemize}
    \end{columns}
\end{frame}

\begin{frame}{Overview of Financial Data from Yahoo Finance}
    \begin{itemize}
        \item \textbf{Stock Data (AAPL - Apple Inc.)}
        \begin{itemize}
            \item \textbf{Open, High, Low, Close}
            \item \textbf{Adj Close}
            \item \textbf{Volume}
        \end{itemize}
        \item \textbf{Cryptocurrency Data (ADA-USD - Cardano)}
        \begin{itemize}
            \item \textbf{Open, High, Low, Close}
            \item \textbf{Volume}
        \end{itemize}
        \item \textbf{Mutual Funds (AEPGX - American Funds EuroPacific Growth Fund; AGTHX - American Funds Growth Fund of America)}
        \begin{itemize}
            \item \textbf{NAV}
        \end{itemize}
        \item \textbf{Bonds (AGG - iShares Core U.S. Aggregate Bond ETF)}
        \begin{itemize}
            \item \textbf{Open, High, Low, Close}
            \item \textbf{Volume}
        \end{itemize}
    \end{itemize}
\end{frame}

\section{Navigating Financial Challenges}
\begin{frame}{Navigating Financial Challenges}
    \begin{itemize}
        \item \textbf{Student Debt:} High levels of student loan debt impact the ability to save.
        \item \textbf{Cost of Living:} Rising costs of rent, utilities, and everyday expenses.
        \item \textbf{Inflation Impact:} Inflation erodes the purchasing power of savings.
    \end{itemize}
\end{frame}

\section{Financial Literacy: Essential Concepts}
\begin{frame}{Financial Literacy: Essential Concepts}
        \begin{itemize}
            \item \textbf{Stocks:}
            \begin{itemize}
                \item Equity investments representing ownership in a company.
                \item Shareholders can earn returns through capital gains and dividends.
                \item Example: Purchasing shares of Apple Inc. (AAPL).
            \end{itemize}
            \item \textbf{Bonds:}
            \begin{itemize}
                \item Debt investments where an investor loans money to an entity (corporate or governmental) for a defined period at a fixed interest rate.
                \item Bonds pay periodic interest (coupon payments) and return the principal amount at maturity.
                \item Example: U.S. Treasury bonds.
            \end{itemize}
        \end{itemize}
\end{frame}

            
            
\begin{frame}{Financial Literacy: Essential Concepts (cont.)}
        \begin{itemize}
            \item \textbf{Mutual Funds:}
            \begin{itemize}
                \item Investment vehicles that pool money from many investors to purchase a diversified portfolio of stocks, bonds, or other securities.
                \item Managed by professional fund managers.
                \item Example: Vanguard 500 Index Fund.
            \end{itemize}
            \item \textbf{ETFs (Exchange-Traded Funds):}
            \begin{itemize}
                \item Similar to mutual funds but traded on stock exchanges like individual stocks.
                \item Provide diversification and are typically more cost-effective.
                \item Example: SPDR S&P 500 ETF (SPY).
            \end{itemize}
        \end{itemize}
\end{frame}

\begin{frame}{Financial Literacy: Essential Concepts (cont.)}
        \begin{itemize}
            \item \textbf{Dollar Cost Averaging:}
            \begin{itemize}
                \item Investing a fixed amount of money at regular intervals, regardless of the asset's price.
                \item Helps mitigate the risk of investing a large amount at an inopportune time.
                \item Example: Investing \$500 monthly into an S&P 500 ETF.
            \end{itemize}
            \item \textbf{Compounding Interest:}
            \begin{itemize}
                \item The process where the value of an investment grows exponentially over time due to earning interest on both the principal and the accumulated interest.
                \item The longer the investment period, the greater the compounding effect.
                \item Example: A \$1,000 investment earning 5\% interest annually.
            \end{itemize}
        \end{itemize}
\end{frame}

\section{Time Value of Money (TVM)}
\begin{frame}{Time Value of Money (TVM)}
    \begin{block}{Definition}
        Time Value of Money (TVM) is the concept that money available today is worth more than the same amount in the future due to its potential earning capacity.
    \end{block}
    \begin{itemize}
        \item \textbf{Importance:}
        \begin{itemize}
            \item Influences various financial decisions, such as investing, borrowing, and budgeting.
            \item Critical for understanding how investments grow over time.
        \end{itemize}
        \item \textbf{Key Formula:}
        \begin{align*}
            \text{Future Value (FV)} &= \text{Present Value (PV)} \times (1 + r)^n \\
            \text{Present Value (PV)} &= \frac{\text{Future Value (FV)}}{(1 + r)^n}
        \end{align*}
        \item \textbf{Explanation:}
        \begin{itemize}
            \item $FV$ is the future value of the investment.
            \item $PV$ is the present value of the investment.
            \item $r$ is the annual interest rate (decimal).
            \item $n$ is the number of years.
        \end{itemize}
    \end{itemize}
\end{frame}

\begin{frame}{Practical Application of TVM}
    \begin{itemize}
        \item \textbf{Investment Growth:} Let's consider the stock data provided (e.g., Apple Inc. - AAPL).
        \item \textbf{Historical Data:}
        \begin{itemize}
            \item Open: \$115.75, Close: \$118.28 (Aug 20, 2020)
            \item Calculate the growth rate: 
            \begin{align*}
                \text{Growth Rate} &= \frac{\text{Close} - \text{Open}}{\text{Open}} \\
                &= \frac{118.28 - 115.75}{115.75} \approx 0.0218 \text{ or } 2.18\%
            \end{align*}
        \end{itemize}
        \item \textbf{Future Value Calculation:}
        \begin{itemize}
            \item Assume we invest \$1,000 in AAPL at an annual growth rate of 2.18\%.
            \item Calculate the future value in 5 years:
            \begin{align*}
                FV &= PV \times (1 + r)^n \\
                FV &= 1000 \times (1 + 0.0218)^5 \\
                FV &\approx 1000 \times 1.113 \\
                FV &\approx 1113
            \end{align*}
            \item The investment would grow to approximately \$1,113 in 5 years.
        \end{itemize}
    \end{itemize}
\end{frame}

\begin{frame}{Using TVM for Housing Down Payments}
    \begin{itemize}
        \item \textbf{Saving for a Down Payment:}
        \begin{itemize}
            \item Suppose we need \$50,000 for a down payment in 10 years.
            \item Calculate the present value needed to save annually with a 2.18\% annual return.
            \item Using the Present Value formula:
            \begin{align*}
                PV &= \frac{FV}{(1 + r)^n} \\
                PV &= \frac{50000}{(1 + 0.0218)^{10}} \\
                PV &\approx \frac{50000}{1.243} \\
                PV &\approx 40222
            \end{align*}
            \item We need to save approximately \$40,222 today to reach \$50,000 in 10 years.
        \end{itemize}
        \item \textbf{Annual Savings Calculation:}
        \begin{itemize}
            \item Alternatively, calculate the annual savings required:
            \begin{align*}
                \text{Annual Savings} &= \frac{FV}{\left( \frac{(1 + r)^n - 1}{r} \right)} \\
                \text{Annual Savings} &= \frac{50000}{\left( \frac{(1 + 0.0218)^{10} - 1}{0.0218} \right)} \\
                \text{Annual Savings} &\approx \frac{50000}{11.080} \\
                \text{Annual Savings} &\approx 4514
            \end{align*}
            \item We need to save approximately \$4,514 annually to reach \$50,000 in 10 years.
        \end{itemize}
    \end{itemize}
\end{frame}

\section{Understanding Market Mechanics}
\begin{frame}{Understanding Market Mechanics}
    \begin{itemize}
        \item \textbf{Stock Market:}
        \begin{itemize}
            \item A marketplace where stocks, bonds, and other securities are bought and sold.
            \item Major exchanges include the New York Stock Exchange (NYSE) and NASDAQ.
            \item Prices are determined by supply and demand dynamics.
        \end{itemize}
        \item \textbf{Market Indices:}
        \begin{itemize}
            \item Benchmarks used to measure the performance of a segment of the market.
            \item Examples include the S&P 500, Dow Jones Industrial Average, and NASDAQ Composite.
        \end{itemize}
    \end{itemize}
\end{frame}

\section{Key Investment Strategies}
\begin{frame}{Key Investment Strategies}
    \begin{itemize}
        \item \textbf{Diversification:}
        \begin{itemize}
            \item Spreading investments across different asset classes to reduce risk.
            \item A diversified portfolio may include stocks, bonds, real estate, and commodities.
        \end{itemize}
        \item \textbf{Asset Allocation:}
        \begin{itemize}
            \item Dividing investments among different asset categories based on an investor's risk tolerance, goals, and investment horizon.
            \item Example: A younger investor might have a higher allocation to stocks, while an older investor might focus more on bonds.
        \end{itemize}
        \item \textbf{Risk Management:}
        \begin{itemize}
            \item Identifying, assessing, and prioritizing risks followed by coordinated efforts to minimize, monitor, and control the impact of unfortunate events.
            \item Strategies include diversification, hedging, and using stop-loss orders.
        \end{itemize}
    \end{itemize}
\end{frame}

\section{Digital Age Investing}
\begin{frame}{Digital Age Investing}
    \begin{itemize}
        \item \textbf{Online Trading Platforms:}
        \begin{itemize}
            \item Platforms like Robinhood, E*TRADE, and TD Ameritrade allow individuals to trade stocks and other securities online.
            \item Lower fees and commissions compared to traditional brokers.
        \end{itemize}
        \item \textbf{Robo-Advisors:}
        \begin{itemize}
            \item Automated platforms that provide financial advice or manage investments based on algorithms.
            \item Examples include Betterment and Wealthfront.
            \item Offer low-cost, diversified investment portfolios.
        \end{itemize}
        \item \textbf{Social Trading:}
        \begin{itemize}
            \item Platforms that allow investors to copy the trades of experienced traders.
            \item Examples include eToro and ZuluTrade.
            \item Helps novice investors learn and participate in the market.
        \end{itemize}
    \end{itemize}
\end{frame}

\section{Applying Linear Algebra in Portfolio Allocation}
\begin{frame}{Applying Linear Algebra in Portfolio Allocation}
    \begin{itemize}
        \item \textbf{Purpose:} Leveraging linear algebra to solve complex portfolio allocation problems.
        \item \textbf{Matrix Operations:} Analyzing the covariance matrix of asset returns to optimize portfolio diversification.
        \item \textbf{Efficient Portfolios:} Using the covariance matrix to calculate portfolios that minimize risk for a given return.
        \item \textbf{Formula:}
        \begin{equation*}
            \text{Minimize } \sigma_p = \sqrt{w^T \Sigma w}
        \end{equation*}
        \item \textbf{Explanation:} In this formula, \( \sigma_p \) represents the portfolio's standard deviation (a measure of risk), \( w \) is a vector of asset weights, and \( \Sigma \) is the covariance matrix of asset returns. The goal is to minimize risk while maintaining the desired portfolio weights.
    \end{itemize}
\end{frame}

\section{Exploring Modern Portfolio Theory (MPT)}
\begin{frame}{Exploring Modern Portfolio Theory (MPT)}
    \begin{itemize}
        \item \textbf{Diversification Principle:} Spreading investments to reduce risk while enhancing returns.
        \item \textbf{Efficient Frontier:} Identifying portfolios that offer maximum return for a given level of risk.
        \item \textbf{Formula:}
        \begin{equation*}
            \text{Efficient Frontier} = \max \left( \frac{E(R_p) - R_f}{\sigma_p} \right)
        \end{equation*}
        \item \textbf{Explanation:} This formula calculates the Sharpe ratio, where \( E(R_p) \) is the expected return of the portfolio, \( R_f \) is the risk-free rate, and \( \sigma_p \) is the standard deviation of the portfolio's return. A higher Sharpe ratio indicates a better risk-adjusted return.
        \item \textbf{Capital Market Line (CML):} Describes the relationship between risk and return for efficient portfolios.
        \item \textbf{Formula:}
        \begin{equation*}
            E(R_p) = R_f + \frac{\sigma_p}{\sigma_m} \left( E(R_m) - R_f \right)
        \end{equation*}
        \item \textbf{Explanation:} In this formula, \( E(R_p) \) is the expected return of the portfolio, \( \sigma_p \) is the portfolio's standard deviation, \( \sigma_m \) is the market portfolio's standard deviation, and \( E(R_m) \) is the expected return of the market portfolio. This formula helps investors understand how adding risk can potentially increase returns.
    \end{itemize}
\end{frame}

\section{Monte Carlo Simulation in Financial Planning}
\begin{frame}{Monte Carlo Simulation in Financial Planning}
    \begin{itemize}
        \item \textbf{Purpose:} To model the probability distributions of achieving target down payment amounts.
        \item \textbf{Scenario Analysis:} Incorporating various economic variables and market scenarios to simulate outcomes.
        \item \textbf{Example:} Assessing how different investment strategies impact down payment accumulation under varying economic conditions.
        \item \textbf{Formula:}
        \begin{equation*}
            X_t = X_0 \exp \left( \left( \mu - \frac{\sigma^2}{2} \right)t + \sigma W_t \right)
        \end{equation*}
        \item \textbf{Explanation:} Here, \( X_t \) is the value of the investment at time \( t \), \( X_0 \) is the initial investment value, \( \mu \) is the drift rate (average return), \( \sigma \) is the volatility (risk), and \( W_t \) is a Wiener process (random movement). This formula models how investment value evolves over time under uncertainty.
    \end{itemize}
\end{frame}

\section{Risk-Return Analysis Techniques}
\begin{frame}{Risk-Return Analysis Techniques}
    \begin{itemize}
        \item \textbf{Objective:} To evaluate the trade-off between risk and expected return for various investment strategies.
        \item \textbf{Sharpe Ratio:} A key metric to compare risk-adjusted returns of different portfolios.
        \item \textbf{Formula:}
        \begin{equation*}
            \text{Sharpe Ratio} = \frac{E(R_p) - R_f}{\sigma_p}
        \end{equation*}
        \item \textbf{Explanation:} This formula measures the excess return per unit of risk. \( E(R_p) \) is the expected return of the portfolio, \( R_f \) is the risk-free rate, and \( \sigma_p \) is the portfolio's standard deviation. A higher Sharpe ratio indicates a better risk-adjusted return.
        \item \textbf{Application:} Using historical data on stock market performance and bond yields to calculate and compare Sharpe ratios.
    \end{itemize}
\end{frame}

\section{Lifecycle Investing Strategies}
\begin{frame}{Lifecycle Investing Strategies}
    \begin{itemize}
        \item \textbf{Purpose:} To adjust risk exposure over time in line with an investor's changing risk tolerance and investment horizon.
        \item \textbf{Dynamic Asset Allocation:} Transitioning from high-risk to low-risk investments as the investment goal approaches.
        \item \textbf{Example:} A lifecycle investment plan that shifts from equities to bonds and cash equivalents as the target date for a down payment nears.
        \item \textbf{Formula:}
        \begin{equation*}
            \text{Allocation}_t = \frac{1}{T-t} \left( \frac{\alpha}{\beta} \right)
        \end{equation*}
        \item \textbf{Explanation:} In this formula, \( T \) is the total investment period, \( t \) is the current time, \( \alpha \) is the expected return, and \( \beta \) is the risk measure. This approach maximizes returns during the early stages and gradually reduces risk to protect accumulated wealth as the goal date approaches.
    \end{itemize}
\end{frame}

\section{Financial Metrics for Project Evaluation}
\begin{frame}{Financial Metrics for Project Evaluation}
    \begin{itemize}
        \item \textbf{Beta (β):} Measures a stock's volatility relative to the overall market. A beta greater than 1 indicates higher volatility than the market, while a beta less than 1 indicates lower volatility.
        \item \textbf{Alpha (α):} Represents the excess return of an investment relative to the return of a benchmark index. Positive alpha indicates outperformance, while negative alpha indicates underperformance.
        \item \textbf{Capital Asset Pricing Model (CAPM):}
        \begin{itemize}
            \item Describes the relationship between systematic risk and expected return for assets, particularly stocks.
            \item Formula: \( E(R_i) = R_f + \beta_i (E(R_m) - R_f) \)
            \item Explanation: \( E(R_i) \) is the expected return of the investment, \( R_f \) is the risk-free rate, \( \beta_i \) is the beta of the investment, and \( E(R_m) \) is the expected return of the market.
        \end{itemize}
    \end{itemize}
\end{frame}

\section{Conclusion}
\begin{frame}{Conclusion}
    \begin{itemize}
        \item \textbf{Summary:} We have explored optimal investment strategies tailored for Generation Z to accumulate housing down payments, incorporating modern financial theories and data-driven insights.
        \item \textbf{Key Findings:} Identified effective strategies such as diversified portfolios, the application of MPT, and lifecycle investing to navigate the unique financial challenges faced by Generation Z.
        \item \textbf{Future Directions:} Ongoing research will focus on refining these strategies and exploring their practical applications to further assist Generation Z in achieving their homeownership goals.
    \end{itemize}
\end{frame}

\section{Q\&A}
\begin{frame}{Q\&A}
    \begin{itemize}
        \item \textbf{Clarifications:} Please feel free to ask for any clarifications or additional details regarding the presented research and findings.
    \end{itemize}
\end{frame}

\section{References}
\begin{frame}{References}
    \begin{itemize}
        \item Federal Reserve Economic Data (FRED). (n.d.). Retrieved from \url{https://fred.stlouisfed.org/}
        \item Yahoo Finance. (n.d.). Retrieved from \url{https://finance.yahoo.com/}
        \item Robinhood. (n.d.). Retrieved from \url{https://robinhood.com/}
        \item Coinbase. (n.d.). Retrieved from \url{https://www.coinbase.com/}
        \item U.S. Census Bureau. (n.d.). Generation Z data. Retrieved from \url{https://www.census.gov/}
        \item Investment Company Institute. (n.d.). Retrieved from \url{https://www.ici.org/}
        \item Sharpe, W. F. (1966). Mutual Fund Performance. Journal of Business, 39(1), 119-138.
        \item Markowitz, H. (1952). Portfolio Selection. Journal of Finance, 7(1), 77-91.
        \item Merton, R. C. (1973). Theory of Rational Option Pricing. Bell Journal of Economics and Management Science, 4(1), 141-183.
    \end{itemize}
\end{frame}

\end{document}
