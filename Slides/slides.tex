\documentclass{beamer}
\usetheme{Boadilla}
\usecolortheme{dolphin}
\usefonttheme{serif}
\setbeamertemplate{navigation symbols}{}
\setbeamertemplate{caption}[numbered]
\usepackage{graphicx}
\usepackage{amsmath}
\usepackage{hyperref}
\usepackage{booktabs}
\usepackage{multicol}
\usepackage{pgfplots}
\pgfplotsset{compat=1.18}

\title{An Economic Analysis of Optimal Investment Strategies for Accumulating Housing Down Payments}
\author{Frank Paul Longo II}
\date{\today}

\begin{document}

\frame{\titlepage}

\begin{frame}{Roadmap}
    \tableofcontents
\end{frame}

\section{Introduction}
\begin{frame}{Introduction}
  \begin{itemize}
    \textbf{Objective:** Develop investment strategies tailored to different age groups for first-time homebuyers to accumulate funds for down payments. 

    \textbf{Motivation:** Rising housing costs pose a challenge for many first-time homebuyers, particularly younger individuals. This research aims to identify optimal investment strategies that can help people in various age groups save effectively for a down payment, ultimately accelerating their path to homeownership. (Consider adding a statistic about rising housing costs here)

  \end{itemize}
\end{frame}


\section{Why First-time Homebuyers Need This Help}
\begin{frame}{Statistics on Typical First-time Homebuyer}
  \begin{itemize}
    \item **Average Age:** 35 years (as of 2023) [Source: National Association of Realtors (NAR)]
    \item **Median Income:** \$95,900 (as of 2023) [Source: Self Financial]
    \item **Marital Status:**
        * 59% Married Couples [Source: National Association of Realtors (NAR)]
        * 19% Single Females
        * 10% Single Males
        * 9% Unmarried Couples
    \item **Challenges:**
        * High student loan debt (national average: \$37,172 in 2023) [Source: EducationData.org]
        * Rising housing costs exceeding income growth
        * Difficulty saving a down payment
    \item **Average Home Cost:** \$348,000 (2022 data) [Source: The Zebra] 
    \item **Down Payment:**  
        * **Average Saved:** \$8,220 (6% of average home price in 2023) [Source: Self Financial]
        * **Typical Percentage:** 8% (may vary depending on loan type) [Source: National Association of Realtors (NAR)]
  \end{itemize}
\end{frame}


\section{Data Sources and Analysis}
\begin{frame}{Data Sources Utilized}
    \begin{itemize}
        \item \textbf{Yahoo Finance (YFinance):}
        \begin{itemize}
            \item Comprehensive financial data on stocks, bonds, and other investment vehicles.
        \end{itemize}
        \item \textbf{Federal Reserve Economic Data (FRED):}
        \begin{itemize}
            \item Essential economic indicators such as inflation rates, interest rates, and unemployment statistics.
        \end{itemize}
    \end{itemize}
\end{frame}

\begin{frame}{Overview of Economic Indicators Data from FRED}
  \begin{itemize}
    \item Consumer Price Index (CPI) 
    \item Producer Price Index (PPI) 
    \item Unemployment Rate 
    \item Gross Domestic Product (GDP) Growth 
    \item Interest Rates 
    \item Housing Starts
    \item Existing Home Sales 
  \end{itemize}
\end{frame}


\begin{frame}{Overview of Financial Data from Yahoo Finance}
    \begin{itemize}
        \item \textbf{Stocks}
        \item \textbf{Cryptocurrency}
        \item \textbf{Mutual Funds}
        \item \textbf{ETFs}
    \end{itemize}
\end{frame}

\section{Financial Literacy: Essential Concepts}
\begin{frame}{Financial Literacy: Essential Concepts}
    \begin{itemize}
        \item \textbf{Stocks:}
        \begin{itemize}
            \item Equity investments representing ownership in a company.
            \item Example: Purchasing shares of Apple Inc. (AAPL).
        \end{itemize}
        \item \textbf{Cryptocurrency:}
        \begin{itemize}
            \item Digital or virtual currencies that use cryptography for security.
            \item Example: Bitcoin (BTC) and Ethereum (ETH).
        \end{itemize}
    \end{itemize}
\end{frame}

\begin{frame}{Financial Literacy: Essential Concepts (cont.)}
    \begin{itemize}
        \item \textbf{Mutual Funds:}
        \begin{itemize}
            \item Investment vehicles that pool money from many investors to purchase a diversified portfolio of stocks, bonds, or other securities.
            \item Managed by professional fund managers.
            \item Example: Vanguard 500 Index Fund.
        \end{itemize}
        \item \textbf{ETFs (Exchange-Traded Funds):}
        \begin{itemize}
            \item Similar to mutual funds but traded on stock exchanges like individual stocks.
            \item Provide diversification and are typically more cost-effective.
            \item Example: SPDR S&P 500 ETF (SPY).
        \end{itemize}
    \end{itemize}
\end{frame}

\section{Time Value of Money (TVM)}
\begin{frame}{Time Value of Money (TVM)}
  \begin{block}{Definition}
    Time Value of Money (TVM) is the concept that money available today is worth more than the same amount in the future due to its potential earning capacity.

    **Formula:** FV = PV * [1 + (i / n)]^(n x t)

    where:
      * FV = Future Value
      * PV = Present Value
      * i = Interest rate per period
      * n = Number of compounding periods per year
      * t = Number of years
  \end{block}
\end{frame}

\begin{frame}{Using TVM for Housing Down Payments}
    \begin{itemize}
        \item \textbf{Monthly Savings Calculation:}
        \begin{itemize}
            \item Calculate the monthly savings required to reach a target down payment adjusted for inflation (CPI).
            \item \textbf{Formula:}
            \begin{equation*}
                PMT = \frac{FV}{\left( \frac{(1 + r/12)^{12n} - 1}{r/12} \right)}
            \end{equation*}
            \item Example Calculation for Different Time Intervals:
            \begin{itemize}
                \item \textbf{5 Years:}
                \begin{align*}
                    FV &= 78000 \times (1 + \text{CPI})^5 \\
                    PMT &= \frac{FV}{\left( \frac{(1 + 0.0218/12)^{60} - 1}{0.0218/12} \right)} \\
                \end{align*}
                \item \textbf{10 Years:}
                \begin{align*}
                    FV &= 78000 \times (1 + \text{CPI})^{10} \\
                    PMT &= \frac{FV}{\left( \frac{(1 + 0.0218/12)^{120} - 1}{0.0218/12} \right)} \\
                \end{align*}
                \item \textbf{15 Years:}
                \begin{align*}
                    FV &= 78000 \times (1 + \text{CPI})^{15} \\
                    PMT &= \frac{FV}{\left( \frac{(1 + 0.0218/12)^{180} - 1}{0.0218/12} \right)} \\
                \end{align*}
            \end{itemize}
        \end{itemize}
    \end{itemize}
\end{frame}



\section{Theory & Analyses}
\begin{frame}{Modern Portfolio Theory}
    \begin{itemize}
        \item \textbf{Objective:} 
        \item \textbf{Results:} 
    \end{itemize}
\end{frame}



\section{Theory & Analyses}
\begin{frame}{CAPM: Capital Asset Pricing Model}
    \begin{itemize}
        \item \textbf{Objective:} 
        \item \textbf{Results:} 
    \end{itemize}
\end{frame}



\section{Theory & Analyses}
\begin{frame}{ARIMA}
    \begin{itemize}
        \item \textbf{Objective:}
        \item \textbf{Results:} 
    \end{itemize}
\end{frame}



\section{Theory & Analyses}
\begin{frame}{GARCH}
    \begin{itemize}
        \item \textbf{Objective:} 
        \item \textbf{Results:} 
    \end{itemize}
\end{frame}




\section{Theory & Analyses}
\begin{frame}{Monte Carlo Simulation}
    \begin{itemize}
        \item \textbf{Objective:} 
        \item \textbf{Results:} 
    \end{itemize}
\end{frame}
















\section{Conclusion}
\begin{frame}{Conclusion}
    \begin{itemize}
        \item \textbf{Future Directions:} We will explore optimal investment strategies tailored for first-time homebuyers to accumulate housing down payments, incorporating modern financial theories and data-driven insights.
        \item Identified effective strategies include diversified portfolios, the application of MPT, and lifecycle investing to navigate the unique financial challenges faced by first-time homebuyers.
        \item Ongoing research will focus on refining these strategies and exploring their practical applications to further assist first-time homebuyers in achieving their homeownership goals.
    \end{itemize}
\end{frame}

\section{Q\&A}
\begin{frame}{Q\&A}
    \begin{itemize}
        \item \textbf{Clarifications:} Please feel free to ask for any clarifications or additional details regarding the presented research and findings.
    \end{itemize}
\end{frame}

\section{References}
\begin{frame}{References}
    \begin{itemize}
        \item Federal Reserve Economic Data (FRED). (n.d.). Retrieved from \url{https://fred.stlouisfed.org/}
        \item Yahoo Finance. (n.d.). Retrieved from \url{https://finance.yahoo.com/}
        \item Robinhood. (n.d.). Retrieved from \url{https://robinhood.com/}
        \item Coinbase. (n.d.). Retrieved from \url{https://www.coinbase.com/}
        \item Investment Company Institute. (n.d.). Retrieved from \url{https://www.ici.org/}
        \item Sharpe, W. F. (1966). Mutual Fund Performance. Journal of Business, 39(1), 119-138.
        \item Markowitz, H. (1952). Portfolio Selection. Journal of Finance, 7(1), 77-91.
        \item Merton, R. C. (1973). Theory of Rational Option Pricing. Bell Journal of Economics and Management Science, 4(1), 141-183.
        \item National Center for Education Statistics. (2020). Student Loan Debt Statistics. Retrieved from \url{https://nces.ed.gov/}
        \item Federal Reserve Bank of New York. (2020). Household Debt and Credit Report. Retrieved from \url{https://www.newyorkfed.org/microeconomics/hhdc.html}
        \item U.S. Department of Housing and Urban Development. (2020). Housing Market Indicators. Retrieved from \url{https://www.huduser.gov/portal/datasets/hmi.html}
        \item Betterment. (n.d.). Retrieved from \url{https://www.betterment.com/}
        \item Wealthfront. (n.d.). Retrieved from \url{https://www.wealthfront.com/}
        \item eToro. (n.d.). Retrieved from \url{https://www.etoro.com/}
        \item ZuluTrade. (n.d.). Retrieved from \url{https://www.zulutrade.com/}
    \end{itemize}
\end{frame}

\end{document}

