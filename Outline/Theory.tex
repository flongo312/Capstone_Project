\section{Theoretical Model}

This section outlines the theoretical frameworks used in this research, including Modern Portfolio Theory (MPT), the Capital Asset Pricing Model (CAPM), and Behavioral Finance.

\subsection{Modern Portfolio Theory (MPT)}
Introduced by Markowitz (1952), Modern Portfolio Theory focuses on constructing portfolios that optimize the trade-off between risk and return. MPT provides the framework for developing efficient portfolios by balancing risk and return, helping to identify optimal portfolio mixes that maximize return for a given risk level. The efficient frontier, a key concept in MPT, represents the set of portfolios that offer the highest expected return for a defined level of risk. This theory will be used to construct diversified investment portfolios for first-time homebuyers.

\subsection{Capital Asset Pricing Model (CAPM)}
Developed by Sharpe (1964), the Capital Asset Pricing Model (CAPM) provides a method to measure the risk-adjusted return of an investment. CAPM is critical for evaluating and comparing different investment strategies by identifying investments with the best returns relative to their risk. The model takes into account the risk-free rate, the beta of the investment, and the expected market return to calculate the expected return of an asset. The CAPM formula is given by:
\[ E(R_i) = R_f + \beta_i (E(R_m) - R_f) \]
where \(E(R_i)\) is the expected return of the asset, \(R_f\) is the risk-free rate, \(\beta_i\) is the beta of the asset, and \(E(R_m)\) is the expected market return.

\subsection{Lifecycle Investing}
Lifecycle investing is a strategy that adjusts the asset allocation based on the investor's age and risk tolerance. Younger investors typically have a higher risk tolerance and a longer time horizon, allowing them to invest more heavily in equities. As investors age, their risk tolerance decreases, and they shift towards more conservative investments, such as bonds. This approach ensures that the investment strategy aligns with the investor's financial goals and risk tolerance throughout their life.

\subsection{Application to First-Time Homebuyers}
The theoretical models discussed will be applied to develop tailored investment strategies for different age groups of first-time homebuyers. The goal is to create portfolios that optimize the accumulation of funds for a down payment while taking into account the unique financial and behavioral characteristics of each age group.
