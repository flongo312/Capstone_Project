\section{Results}

The results of the empirical analysis are presented in this section, highlighting the effectiveness of different investment strategies for first-time homebuyers.

\subsection{Analysis of Investment Strategies}
The analysis reveals that diversified portfolios, which include a mix of stocks, mutual funds, ETFs, and cryptocurrencies, provide the best returns for first-time homebuyers across different age groups. The performance of these portfolios is evaluated over different time horizons, such as 5, 10, and 15 years.

\subsection{Risk-Adjusted Performance}
The Sharpe Ratio and other risk-adjusted performance metrics indicate that strategies incorporating Modern Portfolio Theory (MPT) and lifecycle investing outperform traditional investment approaches, particularly in balancing risk and return. Portfolios that optimize the trade-off between risk and return are found to be more effective in accumulating funds for down payments.

\subsection{Age Group Comparisons}
The results also show that younger age groups benefit more from higher-risk, higher-return investments, while older age groups achieve better outcomes with more conservative, lower-risk strategies. This finding underscores the importance of tailoring investment strategies to the specific needs and risk tolerances of different age groups. For instance, individuals aged 20-25 may benefit from aggressive growth stocks, while those aged 30-35 may prefer a balanced portfolio with a mix of growth and income assets.

\subsection{Monte Carlo Simulation Outcomes}
Monte Carlo simulations indicate that the optimized portfolios have a high probability of achieving the target down payment within the specified timeframes. The simulations provide a probabilistic assessment of the portfolio's performance, highlighting the likelihood of reaching the financial goal based on historical return distributions.

\subsection{Robustness and Sensitivity Analysis}
Robustness checks confirm that the proposed investment strategies are resilient to variations in key assumptions and parameters. Sensitivity analyses reveal that while some variables have a significant impact on the results, the overall conclusions remain consistent across different scenarios. This enhances confidence in the robustness and applicability of the investment strategies developed.

\subsection{Summary of Findings}
The empirical analysis demonstrates that well-diversified portfolios, tailored to the specific age groups of first-time homebuyers, significantly improve the likelihood of accumulating sufficient funds for a down payment. The integration of CAPM, Sharpe Ratio, MPT, and Monte Carlo simulations provides a comprehensive framework for developing and validating these investment strategies.

\subsection{Policy Implications}
The findings have important implications for financial advisors and policymakers. By promoting tailored investment strategies that account for both financial and behavioral factors, they can better support first-time homebuyers in achieving their homeownership goals. Policies aimed at enhancing financial literacy and providing access to diversified investment options could further improve outcomes for first-time homebuyers.

\subsection{Limitations and Future Research}
While the study provides valuable insights, it also has limitations. The reliance on historical data may not fully capture future market dynamics, and the focus on publicly traded securities excludes other potential investment opportunities. Future research could explore the inclusion of alternative assets and the impact of changing economic conditions on investment strategies. Additionally, further studies could investigate the role of financial education and behavioral interventions in enhancing the effectiveness of these strategies.

