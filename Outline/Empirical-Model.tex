\section{Empirical Specification}

This section describes the empirical model used to analyze the data and identify optimal investment strategies. The model applies statistical and econometric methods to the dataset, utilizing techniques such as CAPM, Sharpe Ratio, Modern Portfolio Theory (MPT), and Monte Carlo simulations.

\subsection{Model Specification}
The empirical model employs a multifactor approach to capture the diverse financial attributes influencing investment returns. Key components of the model include:

\subsubsection{Data Loading and Preprocessing}
Data is collected using Python and the Yahoo Finance API, loaded into a DataFrame, and indexed by date. This ensures that all securities are aligned temporally and can be analyzed collectively.

\subsubsection{CAPM Calculation}
The CAPM model is used to estimate the expected return of each security. The steps involved are:
\begin{enumerate}
    \item Calculate the average return of the market index.
    \item Determine the risk-free rate.
    \item Compute the beta (\(\beta\)) of each security.
    \item Use the CAPM formula:
    \[ E(R_i) = R_f + \beta_i (E(R_m) - R_f) \]
    to estimate the expected return for each security.
\end{enumerate}

\subsubsection{Sharpe Ratio Calculation}
The Sharpe Ratio is calculated to measure the risk-adjusted return of each security. The formula used is:
\[ S = \frac{E(R_i) - R_f}{\sigma_i} \]
where \( E(R_i) \) is the expected return, \( R_f \) is the risk-free rate, and \( \sigma_i \) is the standard deviation of the excess return. Securities are then ranked by their Sharpe Ratios.

\subsubsection{Portfolio Optimization Using MPT}
Using the top-ranked securities, portfolios are constructed and optimized for different age groups. The steps include:
\begin{enumerate}
    \item Define the assets and their expected returns and covariances.
    \item Determine the asset weights to maximize expected return for a given level of risk.
    \item Construct the efficient frontier to visualize optimal portfolios.
\end{enumerate}

\subsubsection{Monte Carlo Simulations}
Monte Carlo simulations are used to test the robustness of the optimized portfolios. The steps are:
\begin{enumerate}
    \item Define the initial investment and annual contributions.
    \item Generate random returns based on historical data.
    \item Simulate portfolio growth over 5, 10, and 15 years.
    \item Analyze the distribution of outcomes to evaluate the probability of reaching the target down payment.
\end{enumerate}

\subsection{Model Validation and Robustness Checks}
To ensure the reliability and validity of the empirical model, various robustness checks are conducted. These include:
\begin{itemize}
    \item Testing the sensitivity of the results to different assumptions and parameters.
    \item Verifying the consistency of the CAPM and Sharpe Ratio calculations.
    \item Assessing the impact of data preprocessing and cleaning on the results.
\end{itemize}
These checks help to confirm the robustness of the proposed investment strategies.
