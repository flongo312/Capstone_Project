\section{Literature Review}
\subsection{Modern Portfolio Theory (Markowitz, 1952)}
Harry Markowitz’s Modern Portfolio Theory (MPT) revolutionized the way investors approach portfolio construction by introducing the concept of diversification to optimize the trade-off between risk and return. MPT posits that an efficient portfolio is one that offers the maximum possible return for a given level of risk, or equivalently, the minimum risk for a given level of return. This theory underpins the portfolio optimization models used in this study to develop effective investment strategies for down payment accumulation.

\subsection{Sharpe Ratio (Sharpe, 1966)}
William Sharpe’s introduction of the Sharpe Ratio provided a pivotal tool for assessing the risk-adjusted performance of investment portfolios. The Sharpe Ratio is calculated by dividing the difference between the portfolio return and the risk-free rate by the standard deviation of the portfolio returns. This measure allows investors to compare the performance of different portfolios on a standardized basis, adjusting for the risk taken to achieve those returns. This study employs the Sharpe Ratio to evaluate and compare the effectiveness of various investment strategies.

\subsection{Monte Carlo Methods (Boyle, 1977)}
Phelim Boyle’s application of Monte Carlo methods to financial modeling marked a significant advancement in the ability to simulate and understand the behavior of complex financial instruments under uncertainty. Monte Carlo simulations generate random variables based on historical return distributions to model the potential future performance of investments. This approach is crucial for developing probabilistic models that can predict the accumulation of down payments over time, accounting for the inherent uncertainty in financial markets.

